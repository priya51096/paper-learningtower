% !TeX root = RJwrapper.tex
\title{learningtower: an R package for Exploring Standardised Test
Scores Across the Globe}
\author{by Priya Ravindra Dingorkar, Kevin Y.X. Wang, and Dianne Cook}

\maketitle

\abstract{%
An abstract of less than 150 words - Discuss what the paper talks about
with a little introduction.
}

\hypertarget{introduction}{%
\section{Introduction}\label{introduction}}

The Organization for Economic Cooperation and Development
\href{OECD\%20-\%20https://www.oecd.org/about/}{OECD} is a global
organization that aims to create better policies for better lives. Its
mission is to create policies that promote prosperity, equality,
opportunity, and well-being for all.
\href{PISA\%20-\%20https://www.oecd.org/pisa/}{PISA} is one of OECD's
Programme for International Student Assessment. PISA assesses
15-year-old students' potential to apply their knowledge and abilities
in reading, mathematics, and science to real-world challenges. OECD
launched this in 1997, it was initially administered in 2000, and it
currently includes over
\href{https://www.oecd.org/pisa/aboutpisa/pisa-participants.htm}{80
nations}. The PISA study, conducted every three years, provides
comparative statistics on 15-year-olds' performance in reading, maths,
and science. This paper describes how to utilize the
\texttt{learningtower} package, which offers OECD PISA datasets from
2000 to 2018 in an easy-to-use format. This dataset comprises
information on their test results and other socioeconomic factors, as
well as information on their schools, infrastructure and the countries
participating in the program.

\hypertarget{what-is-pisa}{%
\section{What is PISA?}\label{what-is-pisa}}

PISA assesses the extent to which children approaching the end of
compulsory school have learned some of the information and abilities
required for full participation in modern society, notably in maths,
reading, and science. The examination focuses on reading, mathematics,
science, and problem solving. It also assesses students capacity to
replicate information and extrapolate from what they have learned and
apply that knowledge in unexpected circumstances, both inside and
outside of school. This approach reflects the fact that individuals are
rewarded in modern economies not for what they know, but for what they
can accomplish with what they know.

This evaluation which is carried out every three years, assists in
identifying students' development of knowledge and skills throughout the
world, which can provide actionable insights and therefore assist
education policymakers. PISA is well known for its distinctive testing
characteristics, which include policy orientation, an innovative notion
of literacy, relevance to lifelong learning, regularity, and breadth of
coverage. PISA is now used as an assessment tool in many regions around
the world. In addition to OECD member countries, the survey has been or
is being conducted in East, South and Southeast Asia, Central,
Mediterranean and Eastern Europe, and Central Asia, The Middle East,
Central and South America and Africa.

For each year of the PISA study, one domain subject is thoroughly
examined. In 2018, for example, reading was assessed alongside
mathematics and science as minor areas of assessment. The 2012 survey
concentrates on mathematics, with reading, science, and problem solving
serving as minor evaluation topics. PISA targets a certain age group of
students in order to properly compare their performance worldwide. PISA
students are aged between 15 years 3 months and 16 years 2 months at the
time of the assessment, and have completed at least 6 years of formal
schooling. They can enroll in any sort of institution, participate in
full-time or part-time education, academic or vocational programs, and
attend public, private, or international schools inside the country.
Using this age across nations and throughout time allows PISA to compare
the knowledge and abilities of people born in the same year who are
still in school at the age of 15, irrespective of their diverse
schooling.

The PISA test is primarily computer-based and lasts around 2 hours. The
examination comprises both multiple choice and free entry questions.
Some countries that were not ready for computer-based delivery carried
out the testing on paper. Each student may have a unique set of
questions. An example of the test may be seen
\href{https://www.oecd.org/pisa/test/}{here}. PISA assessment areas seek
to measure the following aspects of students' literacy in math, reading,
and science. The goal of mathematical literacy is to assess students
ability to grasp and interpret mathematics in a variety of settings.
Reading literacy assesses students' capacity to absorb, apply, analyze,
and reflect on texts in order to attain required goals and participate
in society. Science literacy is described as the ability to engage with
science-related issues and scientific concepts as a reflective citizen.

PISA data is publicly accessible for
\href{https://www.oecd.org/pisa/data/}{download}. Furthermore, reading
the
\href{https://www.oecd.org/pisa/data/pisa2018technicalreport/Ch.09-Scaling-PISA-Data.pdf}{data
documentation} reveals that the disclosed PISA scores are generated
using a sophisticated linear model applied to the data. For each
student, several values are simulated. This is known as synthetic data,
and it is a popular technique to ensuring data privacy. The data can
still be deemed accurate within the mean, variance, and stratum used in
the original data's modelling. In addition, the PISA website provides
the data in SPSS and SAS format, which can limit accessibility due to
the commercial nature of these software. Furthermore, all questions are
assigned with unique IDs within each year of the PISA study, but do not
always agree across the different years. This data has now been curated
and simplified into a single R package called \texttt{learningtower},
which contains all of the PISA scores from the years 2000 to 2018.

\hypertarget{data-compilation}{%
\section{Data Compilation}\label{data-compilation}}

Each developer at the ROpenSci OzUnconf was assigned to curate a
specific year of the PISA study. Data on the participating students and
schools were first downloaded from the PISA website, in either SPSS or
SAS format. The data were read into an R environment with the exception
of the year 2000 and 2003. Due to formatting issues, the data for these
two particular years were first read using SPSS and then exported into
compatible \texttt{.sav} files. After some data cleaning and wrangling
with the appropriate script, the variables of interest were
re-categorised and saved as RDS files. One major challenge faced by the
developers was to ensure the consistency of variables over the years.
For example, a student's mother's highest level of education was never
recorded in 2000, but it was categorised as ``ST11R01'' between 2003 and
2012 and ``ST005Q01TA'' between 2015 and 2018. Such a problem was
tackled manually by curating these values as an integer variable named
``mother\_educ'' in the output data. These final RDS file for each PISA
year were then thoroughly vetted and made available in a separate
\href{https://github.com/kevinwang09/learningtower_masonry}{GitHub
repository}.

\hypertarget{what-is-learningtower}{%
\section{\texorpdfstring{What is
\texttt{learningtower}?}{What is learningtower?}}\label{what-is-learningtower}}

\href{https://cran.r-project.org/web/packages/learningtower/index.html}{`learningtower'}
is an easy-to-use R package that provides quick access to a variety of
variables using OECD PISA data collected over a three-year period from
2000 to 2018. This dataset includes information on the PISA test scores
in mathematics, reading, and science. Furthermore, these datasets
include information on other socioeconomic aspects, as well as
information on their school and its facilities, as well as the nations
participating in the program.

The motivation for developing the \texttt{learningtower} package was
sparked by the announcement of the PISA 2018 results, which caused a
collective wringing of hands in the Australian press, with headlines
such as
\href{https://theconversation.com/vital-signs-australias-slipping-student-scores-will-lead-to-greater-income-inequality-128301}{``Vital
Signs: Australia's slipping student scores will lead to greater income
inequality''} and
\href{https://www.smh.com.au/education/in-china-nicholas-studied-maths-20-hours-a-week-in-australia-it-s-three-20191203-p53ggv.html}{``In
China, Nicholas studied maths 20 hours a week. In Australia, it's
three''}. That's when several academics from Australia, New Zealand, and
Indonesia decided to make things easier by providing easy access to PISA
scores as part of the \href{https://ozunconf19.ropensci.org/}{ROpenSci
OzUnconf}, which was held in Sydney from December 11 to 13, 2019. The
data from this survey, as well as all other surveys performed since the
initial collection in 2000, is freely accessible to the public. However,
downloading and curating data across multiple years of the PISA study
could be a time consuming task. As a result, we have made a more
convenient subset of the data freely available in a new R package called
\texttt{learningtower}, along with sample code for analysis.

The \texttt{learningtower} package primarily comprised of three
datasets: \texttt{student}, \texttt{school}, and \texttt{countrycode.}
The \texttt{student} dataset includes results from triennial testing of
15-year-old students throughout the world. This dataset also includes
information about their parents' education, family wealth, gender, and
presence of computers, internet, vehicles, books, rooms, desks, and
other comparable factors. Due to the size limitation on CRAN packages,
only a subset of the student data can be made available in the
downloaded package. These subsets of the student data, known as the
\texttt{student\_subset\_yyyy} (\texttt{yyyy} being the specific year of
the study) allow uses to quickly load, visualise the trends in the full
data. The full student dataset can be downloaded using the
\texttt{load\_student()} function included in this
\href{https://kevinwang09.github.io/learningtower/}{package.} The
\texttt{school} dataset includes school weight as well as other
information such as school funding distribution, whether the school is
private or public, enrollment of boys and girls, school size, and
similar other characteristics of interest of different schools these
15-year-olds attend around the world. The \texttt{countrycode} dataset
includes a mapping of a country/region's ISO code to its full name.

\texttt{learningtower} developers are committed to providing R users
with data to analyse PISA results every three years. Our package's
future enhancements include updating the package every time additional
PISA scores are announced. Note that, in order to account for post
COVID-19 problems, OECD member nations and associates decided to
postpone the PISA 2021 evaluation to 2022 and the PISA 2024 assessment
to 2025.

\hypertarget{example-analysis}{%
\section{Example Analysis}\label{example-analysis}}

In this section we will illustrate how the \texttt{learningtower}
package can be utilized to answer some research questions by applying
various methodologies and statistical computations on the
\texttt{learningtower} datasets.

We will solely utilize the 2018 PISA data and scores for illustrative
purposes throughout the example analysis section. During the
post-development phase, the \texttt{learningtower} developers
collectively decided to answer a few intriguing questions on the PISA
data and see if we could identify any interesting trends or insights
utilizing this dataset. Some of these questions include if there is any
significant gender difference between girls and boys whos perform is
better in any of the three areas of mathematics, reading, and science.
Do the various socioeconomic characteristics reflected in the student
data have a substantial impact on the scores of these 15-year-olds.
Furthermore, we will delve into Australia's score history and temporal
trend to uncover some noteworthy trends that Australia has observed as a
result of its participation in the PISA experiment.

Gender gaps have always been a topic of interest among researchers, and
when it comes to PISA data and scores of 15-year-old students around the
world, uncovering patterns based on their gender would help gain
meaningful insights in the field of education for various education
policymakers around the world. Based on the 2018 PISA results, let us
see if there is a major gender disparity between girls and boys
throughout the world in mathematics, reading, and science. To begin, we
will create a `data.frame' that stores the weighted average maths score
for each nation as well as the various regions of the countries
organized by country gender.
\href{https://www.oecd.org/pisa/data/2015-technical-report/PISA-2015-Technical-Report-Chapter-8-Survey-Weighting.pdf}{Survey
weights} are critical and must be used in the analysis to guarantee that
each sampled student accurately represents the total number of pupils in
the PISA population. In addition, we compute the gender difference
between the two averages. To demonstrate the variability in the mean
estimate, we use bootstrap sampling with replacement on the data and
compute the same mean difference estimate. For each nation, the
empirical 90 percent confidence intervals are presented. The same
process is used for reading and science test scores.

\hypertarget{gender-analysis}{%
\section{Gender Analysis}\label{gender-analysis}}

\begin{Schunk}
\begin{figure}[H]
\includegraphics[width=1\linewidth]{learningtower_files/figure-latex/score-differences-1} \caption[The chart above depicts the gender gap difference in 15-year-olds' in math, reading, and science results in 2018]{The chart above depicts the gender gap difference in 15-year-olds' in math, reading, and science results in 2018. The scores to the right of the red line represent the performances of the girls, while the scores to the left of the red line represent the performances of the boys. One of the most intriguing conclusions we can get from this chart is that in the PISA experiment in 2018, girls from all nations outperformed boys in reading.}\label{fig:score-differences}
\end{figure}
\end{Schunk}

Figure \ref{fig:score-differences} illustrates the global disparities in
mean math, reading, and science outcomes Before we get to the plot
conclusion, let's have a look at the variables that have been plotted.
The red line here indicates a reference point, and all of the scores to
the right of the red line show the scores of girls in math, reading, and
science. Similarly, the scores on the left side of this line indicate
the scores of boys in the three disciplines. Based on figure
\ref{fig:score-differences}, because most math estimates and confidence
intervals lie to the left of the red line, we may conclude that most
boys outperformed girls in math. In nations such as Morocco,
Netherlands, Slovenia, Poland, Bulgaria, and Greece, there is almost no
gender difference in math outcomes. When we look at the reading scores,
we notice a really interesting detail: girls outpaced boys in reading in
all countries in 2018. The highest reading scores were achieved by girls
from Qatar, the United Arab Emirates, and Finland. Looking further into
the science plot, we see an unexpected pattern: most nations have very
little gender difference in science scores, implying that most boys and
girls perform equally well in science. Boys from Peru, Colombia, and
regions of China perform really well in science and girls from Qatar,
the United Arab Emirates, and Jordan are the top scores for science.
Figure \ref{fig:score-differences} helps us depicts the gender gap in
math, reading, and science for all nations and regions that took part in
the 2018 PISA experiment.

We gathered meaningful insights about the gender gap between girls and
boys throughout the world from the above figure
\ref{fig:score-differences} because this is a geographical research
communication topic, the findings will help us better comprehend the
score differences in the three educational disciplines using globe maps.
Let us continue to investigate and discover patterns and correlations
using this, map visualisation. To illustrate the gender gap difference
between girls and boys throughout the world, we utilize the
\texttt{map\_data} function to get the latitude and longitude
coordinates needed to construct a map for our data. We connect these
latitude and longitude coordinates to our PISA data and render the world
map using \texttt{geom\_polygon} function wrapped within
\texttt{ggplot2}.

\begin{Schunk}
\begin{figure}[H]
\includegraphics[width=1\linewidth]{learningtower_files/figure-latex/ggplot-maps-1} \caption[Maps that show the gender gap in math, reading, and science results between girls and boys throughout the world]{Maps that show the gender gap in math, reading, and science results between girls and boys throughout the world. The diverging colour scale makes it possible to interpret the range of scores and the also helps us intrepret the gender gap difference among these students across the globe. The legend for each discipline enables interpretation of the score differential for each subject across all maps. A positive score for a country indicates that girls outperformed boys in that country, whereas a negative score for a country difference indicates that boys outperformed girls in that country.The reading scores are all positive, suggesting that girls outperform boys globally in the year 2018.}\label{fig:ggplot-maps}
\end{figure}
\end{Schunk}

In the graphic \ref{fig:ggplot-maps}, we have used maps to show the
gender gap difference between girls and boys in math, reading, and
science in 2018. Map visualization aids in the comprehension of large
volumes of data at a look and in a more efficient manner. Increases the
ability to compare outcomes across many geographies at a glance. In the
illustration, we see both positive and negative score difference scale
ranges in all three maps. A positive country score indicates that girls
outperformed boys in that country, whereas a negative country score
shows that boys outscored girls in that country. Observing the legends
of these maps further helps us deduce the large gender discrepancy
between girls and boys. Looking at these three maps together, we can see
that the most gender difference can be seen in math and reading scores,
where most boys around the world outperform girls in the subject of
mathematics, whereas all positive reading scores indicate that all girls
around the world outperformed boys in reading in 2018. Furthermore,
there is less of a gender disparity in science scores. Furthermore, the
separate diverging scale for each subject aids in the interpretation of
the gender gap score transferred to different nations. It also assists
us in identifying the nations that took part in the PISA experiment. The
grey colour for different geographic locations across the maps in figure
\ref{fig:ggplot-maps} indicates that these regions were not a part of
the PISA experiment in year 2018.

As a result, in this section, we have seen the gender gap scores and
striking trends between 15-year-old girls and boys in math, reading, and
science. Our main conclusion from this gender study is the performance
of girls in reading. The fewer gender disparity evident in science
scores, and the majority of boys perform better than girls in
mathematics.

\hypertarget{socioeconomic-factors-analysis}{%
\section{Socioeconomic Factors
Analysis}\label{socioeconomic-factors-analysis}}

Socioeconomic status is an economic and sociological complete measure of
a person's work experience, economic access to resources, and social
standing in relation to others. Do these socioeconomic factors influence
students' academic performance? In this section, we will investigate if
different socioeconomic factors owned by a family have a significant
impact on a student's academic performance. The student dataset in the
\texttt{learningtower} package is made up of scores from triennial
testing of 15-year-olds throughout the world. This dataset also includes
information about their parents' education, family wealth, gender, and
ownership of computers, internet, cars, books, rooms, desks, and
dishwashers. In this section, we will mainly explore intriguing figures
and derive some fascinating aspects of the influence of a few
socioeconomic factors on student performance in math, reading, and
science. Before we go on to our socioeconomic determinants and their
impact on students, figure \ref{fig:corr-plot} shows how the math,
reading, and science scores are strongly positively related to one
another.

\begin{Schunk}
\begin{figure}[H]
\includegraphics[width=1\linewidth]{learningtower_files/figure-latex/corr-plot-1} \caption[The scatterplot displays the relationship between math, reading, and science scores for all PISA countries that participated in the experiment in 2018]{The scatterplot displays the relationship between math, reading, and science scores for all PISA countries that participated in the experiment in 2018. This scatterplot shows that all three subjects have a significant and positive correlation with one another.}\label{fig:corr-plot}
\end{figure}
\end{Schunk}

We plotted all three scores against each other in the figure
\ref{fig:corr-plot} to understand the relationship between them. This
data allows us to deduce that math, science, and reading scores are
positively and significantly connected with one another, implying that
when we inspect and analyze the many socioeconomic determinants, we
would get comparable outcomes with each score and influence of desire
component. As a result of seeing this substantially positive association
between all three scores for all countries that participated in the PISA
experiment in 2018, the \texttt{learningtower} developers decided to
show the effect of socioeconomic variables on average math scores only
because of the significant association between present reading and
science scores would tend to show similar patterns as well. Let us
further explore the impact of the a few socioeconomic factors on the
students score.

Parents qualification is and will always be a vital element of childhood
development. As previously stated, the student dataset in the package
includes information regarding the parents qualification. In this
section, we will investigate if both the mother's and father's
qualifications have a significant impact on the child's performance. The
mother and father qualifying variables are originally recorded in the
package at distinct levels that are less than ISCED1 equivalent to ISCED
0, ISCED 1, ISCED 2, ISCED 3A and ISCED 3B, C. The International
Standard Classification of Education, (ISCED) who decides and classifies
these levels, where level 0 indicates pre-primary education or no
education at all, level 1 indicates primary education or the first stage
of basic education, level 2 indicates lower secondary education or the
second stage of basic education, and level 3 indicates upper secondary
education. ISCED level 3 have been further classified into three
distinct levels, with ideally very little difference in their
classification. This may also be found in the publication
\href{https://www.oecd.org/education/1841854.pdf}{Classifying
Educational Programmes} published by ISCED. To determine the impact of
parents qualification we first create a \texttt{data.frame} that is
first categorized by the various countries and regions and grouping by
the father's amd mother qualification as well. We next compute the
average weighted weighted mean of math scores while accounting for
student weights. Furthermore, we restructured this variable based on the
multiple levels of classification, dividing it into four unique levels
of education, namely early childhood, primary, lower, and secondary
education. Furthermore, we display the weighted math average versus
qualification for both the mother and father using the
\texttt{geom\_quasirandom} function wrapped within \texttt{ggplot2.}

\begin{Schunk}
\begin{figure}[H]
\includegraphics[width=1\linewidth]{learningtower_files/figure-latex/qual-plot-1} \caption[The influence of parents' education on their children's academic success]{The influence of parents' education on their children's academic success. When the parents have attained higher levels of education, the figure shows a significant increase in scores and an increase in the median of scores for each category. When compared to the parents who have lesser levels of education qualifications. Parents upper secondary education or equivalent qualifications children tend to scoree higer in weighted average math scores.}\label{fig:qual-plot}
\end{figure}
\end{Schunk}

The figure \ref{fig:qual-plot} depicts the impact of mothers' and
fathers' qualifications on student academic performance. The figure
\ref{fig:qual-plot} allows us to deduce a very important and remarkable
insight in which we see a constant increase in the students' academic
performance when both mother and father qualifications shift towards
higher levels of education. The bold horizontal black lines that we see
in each category for mother's and father's qualification here represent
the category's median score. As the parent attains better
qualifications, we notice an increasing trend in these medians for each
category. Taking a closer look at the figure \ref{fig:qual-plot}, we can
see that there is a considerable boost in scores when both the mother
and father have upper secondary education. Furthermore, utilizing
\texttt{the\ geom\_quasirandom} function makes this plot more accessible
and understandable by providing a way to offset points inside categories
to prevent overplotting.

\hypertarget{tv-plots}{%
\subsection{TV Plots}\label{tv-plots}}

\begin{Schunk}
\begin{figure}[H]
\includegraphics[width=1\linewidth]{learningtower_files/figure-latex/tv-plot-1} \caption[TV Plot]{TV Plot}\label{fig:tv-plot}
\end{figure}
\end{Schunk}

Figure \ref{fig:tv-plot}

\hypertarget{book-plot}{%
\subsection{Book Plot}\label{book-plot}}

\begin{Schunk}
\begin{figure}[H]
\includegraphics[width=1\linewidth]{learningtower_files/figure-latex/book-plot-1} \caption[Book Plot]{Book Plot}\label{fig:book-plot}
\end{figure}
\end{Schunk}

Figure \ref{fig:book-plot}

\hypertarget{internet-and-computer-plot}{%
\subsection{Internet and Computer
Plot}\label{internet-and-computer-plot}}

\begin{Schunk}
\begin{figure}[H]
\includegraphics[width=1\linewidth]{learningtower_files/figure-latex/compint-plot-1} \caption[Qualification Plots]{Qualification Plots}\label{fig:compint-plot}
\end{figure}
\end{Schunk}

Figure \ref{fig:compint-plot}

\hypertarget{temoral-trend-australia}{%
\section{Temoral Trend Australia}\label{temoral-trend-australia}}

\bibliography{learningtower.bib}

\address{%
Priya Ravindra Dingorkar\\
Monash University\\%
Department Econometrics and Business Statistics\\ Clayton, Australia\\
%
\url{https://www.linkedin.com/in/priya-dingorkar/}\\%
%
\href{mailto:priyadingorkar@gmail.com}{\nolinkurl{priyadingorkar@gmail.com}}%
}

\address{%
Kevin Y.X. Wang\\
University of Sydney\\%
Data scientist\\ Illumina, Inc.\\ School of Mathematics and
Statistics\\ Sydney, Australia\\
%
\url{https://kevinwang09.github.io/}\\%
%
\href{mailto:kevinwangstats@gmail.com}{\nolinkurl{kevinwangstats@gmail.com}}%
}

\address{%
Dianne Cook\\
Monash University\\%
Department Econometrics and Business Statistics\\ Clayton, Australia\\
%
\url{http://dicook.org/}\\%
%
\href{mailto:dicook@monash.edu}{\nolinkurl{dicook@monash.edu}}%
}
